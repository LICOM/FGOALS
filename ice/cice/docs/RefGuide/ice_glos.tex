\section{Glossary}

% USAGE NOTE:
%
% The first use of a term in the text of a document can be 
% linked to the corresponding glossary item using the item label.
% 
% For example,
%
% Original document text: code must include item1
%
% Linked to glossary:     code must include \htmlref{item1}{glos:item1}
%
% The link will appear in the html version of the document.
% The print version of the document will appear unchanged.

\begin{description}

\item [B component set] The shorthand name for the combination of all
                        active components of CCSM:  ocean, atmosphere,
                        land, and ice exchanging information via a coupler.

\item [CCSM] The Community Climate System Model is a fully-coupled,
             global climate model that provides state-of-the-art computer
             simulations of the Earth's past, present, and future climate
             states.

\item [CICE] The Los Alamos sea ice model

\item [CSIM] The CCSM Community Sea Ice Model 

\item [CSIM4] Version 4 of the CCSM Community Sea Ice Model that
              was released in May 2002 with CCSM2.0.

\item [CSIM5] Version 5 of the CCSM Community Sea Ice Model that
              was released in June 2004 with CCSM3.0.

\item [CVS]  The Concurrent Versions System used to record the
             history of source code.

\item [CVS branch] A line of development separate from the main trunk.

\item [CVS main trunk] The main line of development
                        on the CVS source tree.

\item [inline] An optimization where the code of called routines
               is inserted into the calling code to eliminate the calling overhead.

\item [LANL] Los Alamos National Laboratory

\item [load balancing]  The distribution of processing and communications
                        activity between components to minimize the resources used and the
                        time spent waiting by any single component. 

\item [M component set] The shorthand name for the combination of 
                        data ocean, atmosphere, and land models,
                        csim with the mixed layer ocean, exchanging
                        information via a coupler.

\item [MPI] Message Passing Interface is the protocol for passing messages
                             between parallel processors.

\item [NCAR] National Center for Atmospheric Research, in Boulder, Colorado

\item [PCWG] The Polar Climate Working Group is a team of scientists
             who develop and improve the sea ice component of CCSM,
             and who use the ice model alone or fully coupled in
             CCSM for studies of polar climate.

\item [ProTeX]  A perl script  that allows for automatic generation of
                Latex compatible documentation of source code, without a
                considerable effort beyond the documentation of the code itself. 

\end{description}








